\documentclass[uplatex,a4paper,12pt]{jsarticle}

%% フォント
%\usepackage{newtxtext,newtxmath}
%% 図、文字色
\usepackage[dvipdfmx]{graphicx,xcolor}
%% ハイパーリンク
\usepackage[dvipdfmx]{hyperref}
%% しおりの日本語
\usepackage{pxjahyper}
%% しおりに節番号を含める、リンクの枠を出力しない
\hypersetup{
  bookmarksnumbered=true,
  hidelinks=true,
  colorlinks=false,
}
%% 特殊記号
\usepackage{textcomp}
%% URL
\usepackage{url}
%% 表
%\usepackage{multirow}
%% 参照漏れチェック
%\usepackage{refcheck}

%図と表の表記
%\renewcommand{\figurename}{Fig.}
%\renewcommand{\tablename}{Table.}

\usepackage[top=30truemm,bottom=30truemm,left=30truemm,right=30truemm]{geometry}

\begin{document}

% 表紙
\begin{picture}(420,520)(-20,20)
%契約番号(必須)
\put(330,470){\makebox(50,5)[l]{\normalsize{2023情財第 XXX 号}}} %「XXX」はプロジェクトごとの番号に置き換える
%事業名(必須)
\put(140,350){\makebox(100,15)[c]{\LARGE{2023年度未踏IT人材発掘・育成事業}}}
%プロジェクト名
\put(140,320){\makebox(100,15)[c]{\LARGE{ぬいぐるみ専用の組み込みAIモジュールの開発}}}
% \put(140,290){\makebox(100,15)[c]{\LARGE{契約名2行目(必要なら)}}}
\put(140,260){\makebox(100,15)[c]{\LARGE{成果報告書}}} %2行目がなければ\put(140,290)
%クリエータ名1(必須、人数に応じて\putの位置を調整)
\put(140,140){\makebox(100,10)[c]{\Large{クリエータ:小山 高}}}
%クリエータ名2(必須、人数に応じて\putの位置を調整)
% \put(140,110){\makebox(100,10)[c]{\Large{      姓 名}}}
% 担当PM名(クリエータ2がいなければ\put(140,100)
\put(140,70){\makebox(100,10)[c]{\Large{担当PM:稲見 昌彦}}}
%日付(必須、通常は契約最終日、クリエータ2がいなければ\put(140,60)
\put(140,30){\makebox(100,15)[c]{\Large{2024年3月8日}}}
\end{picture}
\thispagestyle{empty}
\clearpage

\tableofcontents
\thispagestyle{empty}
\clearpage

% 本文
\pagenumbering{arabic}
\setcounter{page}{1}

\section{要約}

\section{背景及び目的}

\section{プロジェクト概要}

\section{開発内容}

\section{開発成果の特徴}

\section{今後の課題、展望}

\section{実施計画書内容との相違点}

\section{開発分担}

\section{成長の自己分析}

\section{秘匿ノウハウの指定}

\section{その他}

\section{付録}

\subsection{用語説明}

\subsection{関連Webサイト}

\end{document}
