\documentclass[uplatex,a4paper,12pt]{jsarticle}

%% フォント
%\usepackage{newtxtext,newtxmath}
%% 図、文字色
\usepackage[dvipdfmx]{graphicx,xcolor}
%% ハイパーリンク
\usepackage[dvipdfmx]{hyperref}
%% しおりの日本語
\usepackage{pxjahyper}
%% しおりに節番号を含める、リンクの枠を出力しない
\hypersetup{
  bookmarksnumbered=true,
  hidelinks=true,
  colorlinks=false,
}
%% 特殊記号
\usepackage{textcomp}
%% URL
\usepackage{url}
%% 表
%\usepackage{multirow}
%% 参照漏れチェック(デバッグ用)
%\usepackage{refcheck}

% キャプションとページ番号をゴシック体に
\renewcommand{\kanjifamilydefault}{\gtdefault}
\usepackage{caption}
\usepackage{fancyhdr,lastpage}
\fancypagestyle{foot}
{
  \lhead{}
  \chead{}
  \rhead{}
  \cfoot{{\sf \thepage{}/{}\pageref{LastPage}}}
  \renewcommand{\headrulewidth}{0.0pt}
}

\usepackage[top=30truemm,bottom=30truemm,left=30truemm,right=30truemm]{geometry}
\usepackage{here}
\usepackage{fancyhdr}
\usepackage{lastpage}
\fancypagestyle{mypagestyle}{%
\lhead{}%ヘッダ左を空に
\rhead{}%ヘッダ右を空に
\cfoot{\thepage/\pageref{LastPage}}%フッタ中央に"今のページ数/総ページ数"を設定
\renewcommand{\headrulewidth}{0.0pt}%ヘッダの線を消す
}
\pagestyle{mypagestyle}
\begin{document}

% http://www.latex-cmd.com/style/style.html
% \gt % ゴシック体
% \sf % サンセリフ体
\gtfamily\sffamily

\makeatletter
\renewcommand{\section}{\@startsection{section}{1}{\z@}{1.5\Cvs \@plus.5\Cvs \@minus.2\Cvs}{.5\Cvs \@plus.3\Cvs}{\reset@font\bfseries\gtfamily\sffamily}}
\makeatother
\captionsetup[figure]{font=sf}
\pagestyle{foot}

\pagenumbering{arabic}
\setcounter{page}{1}

% タイトルとサブタイトル(ad-hoc)
%% maketitleでタイトルを入れたければ、プリアンブルを調整して同じ様に表示されるようにしてください。
\begin{center}
\fontsize{14pt}{0pt}\selectfont
ぬいぐるみ専用の組み込みAIモジュールの開発\\
% - サブタイトル - 
\end{center}
    

\section{背景}

\section{目的}

\section{開発の内容}

\section{従来の技術との相違}

\section{期待される効果}

\section{普及の見通し}
%\section{活用の見通し}
% テンプレートでは「普及(または活用)の見通し」になっていますが、どちらか、あるいは両方を選んで記述をして、見出しをそれに合わせてください。
% 目安です。
% 「普及の見通し」→ユーザ獲得が先にある場合
% 「活用の見通し」→秘匿性が高く自分自身で活用(例えば起業)することが先にある場合
% 「普及・活用の見通し」→両方

\section{小山 高(東京大学大学院 情報理工学系研究科 知能機械情報学専攻 修士2年)}
\begin{itemize}
  \item Aさんの氏名(A)
\end{itemize}

\section*{(参考)関連URL} % もしあれば
% 関連URLは参考文献のリンクではなくプロジェクト・プロダクトに直接関係するものへのリンクです。成果詳細では参考文献を貼るのはなしです。
\begin{itemize}
  \item ティザーサイト:\url{https://pages.github.com/}
  \item ソースコード:\url{https://github.com/}
\end{itemize}

\end{document}
